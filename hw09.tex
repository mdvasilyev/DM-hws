%! suppress = Makeatletter
%! suppress = TooLargeSection
%! suppress = MissingLabel
\documentclass{article}

% Fields
\usepackage{geometry}
\geometry{top=25mm}
\geometry{bottom=35mm}
\geometry{left=20mm}
\geometry{right=20mm}
% ------------------------------------------------

% Graphics
\usepackage{color}
\usepackage{tabularx}
\usepackage{tikz}
\usepackage{blkarray}
\usepackage{graphicx}
% ------------------------------------------------

% Math
\usepackage{amsmath, amsfonts}
\usepackage{amssymb}
\usepackage{proof}
\usepackage{mathrsfs}
% Crossed-out symbols
% https://tex.stackexchange.com/questions/75525/how-to-write-crossed-out-math-in-latex
\usepackage[makeroom]{cancel}
\usepackage{mathtools}
% ------------------------------------------------

% Additional font sizes
% https://www.overleaf.com/learn/latex/Questions/How_do_I_adjust_the_font_size%3F
\usepackage{moresize}
% Additional colors
% https://www.overleaf.com/learn/latex/Using_colours_in_LaTeX
\usepackage{xcolor}
% \texttimes
\usepackage{textcomp}
% ------------------------------------------------

% Language
\usepackage[utf8] {inputenc}
\usepackage[T2A] {fontenc}
\usepackage[english, russian] {babel}
\usepackage{indentfirst, verbatim}
\usetikzlibrary{cd, babel}
% ------------------------------------------------

% Fonts
\usepackage{stmaryrd}
\usepackage{cmbright}
\usepackage{wasysym}
% ------------------------------------------------

% Code
% https://tex.stackexchange.com/questions/99475/how-to-invoke-latex-with-the-shell-escape-flag-in-texstudio-former-texmakerx
% Colored, requires --shell-escape compiling option
% \usepackage{minted}
% \setminted{xleftmargin=\parindent, autogobble, escapeinside=\#\#}
\usepackage{listings}
% ------------------------------------------------

% Custom envs
% https://tex.stackexchange.com/questions/371286/draw-a-horizontal-line-in-latex
\newenvironment{proof}{\subparagraph{\hspace{-1em}Решение:\newline}}{\par\noindent\rule{\textwidth}{0.4pt}}
% ------------------------------------------------

% Custom commands
\newcommand{\comb}[1]{\mathbf{#1}}
\newcommand{\step}{\rightsquigarrow}
\newcommand{\term}[1]{\mathbf{#1}}
\newcommand{\ap}{~}
\newcommand{\termdef}{\coloneqq}
\newcommand{\subst}[3]{\left[#2 \mapsto #3 \right] #1}
\newcommand{\eqbeta}{=_\beta}
\newcommand{\eqeta}{=_\eta}
\def\multiset#1#2{\ensuremath{\left(\kern-.3em\left(\genfrac{}{}{0pt}{}{#1}{#2}\right)\kern-.3em\right)}}
% ------------------------------------------------

% Head
\usepackage{fancybox,fancyhdr}
\usepackage{hyperref}
\pagestyle{fancy}
\fancyhead[R]{Максим Васильев (285800)} % TODO введите ваше имя
\fancyhead[L]{ИТМО MSE, ДМ 2023, Дз 9}
% ------------------------------------------------

% Numbering
% https://tex.stackexchange.com/questions/80113/hide-section-numbers-but-keep-numbering
\makeatletter
\renewcommand\thesubsection{Блок \@arabic\c@subsection.\hspace{-0.8em}}
\renewcommand\thesubsubsection{Задание \@arabic\c@subsection.\@arabic\c@subsubsection\hspace{-0.8em}}
% https://tex.stackexchange.com/questions/327689/numbering-subsubsections-with-letters
\renewcommand\theparagraph{\alph{paragraph})\hspace{-0.8em}}
% https://tex.stackexchange.com/questions/129208/numbering-paragraphs-in-latex
\setcounter{secnumdepth}{4}
\makeatother
% ------------------------------------------------

\begin{document}
\begin{enumerate}

\item \textit{(1 балл)} Выясните, существует ли случайная величина $X$ такая, что $EX = 1$, $EX^2 = 2$, $EX^3 = 3$, $EX^4 = 4$ и $EX^5 = 5$.
\item (2 балла) Паша кидает игральную кость и получает $X$ очков, где $X$ — число выпавших очков на кости. После этого Слава подкидывет монетку и получает $Y$ очков, где $Y = 0$, если выпал орел, и $Y = 2X$, если выпала решка. Вычислите коэффициент корреляции $r(X, Y)$.
\item \textit{(2 балла)} На перрон станции метро каждые 5 минут приходит случайное число пассажиров, распределенное по закону Пуассона с параметром $\lambda = 250$. За то же время с перрона отправляются проходящие поезда, которые могут увезти количество пассажиров, имеющее равномерное распределение в промежутке $[195, 205]$. Можно ли рассчитывать, что перрон, вмещающий N человек, не переполнится, если такой режим поддерживается постоянно?
\item \textit{(1 балл)} Пусть ${X_n}$ — последовательность независимых случайных величин, причем $X_k$ принимает значения $-2^k, -1, 1, 2^k$ с вероятностями $2^{-2k-1},\frac{1-2^{-2k}}{2},\frac{1-2^{-2k}}{2}, 2^{-2k-1}$ соответственно. Выполнен ли ЗБЧ для последовательности ${X_n}$?

% \item \textit{(по 1 баллу за пункт)} Игральная кость подбрасывается до тех пор, пока общая сумма выпавших очков не превысит 700. Оцените вероятность того, что для этого потребуется 

% а) более 210 бросаний;

% б) менее 180 бросаний.

% \item \textit{(2 балл)} При выстреле по мишени стрелок попадает в десятку с вероятностью 0.35, в девятку — 0.3, в восьмерку — 0.2, в семерку — 0.1, в шестерку — 0.05. Стрелок сделал 100 выстрелов. Какова вероятность того, что он набрал не менее 900 очков?

\end{enumerate}
\end{document}
