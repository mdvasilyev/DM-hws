%! suppress = Makeatletter
%! suppress = TooLargeSection
%! suppress = MissingLabel
\documentclass{article}

% Fields
\usepackage{geometry}
\geometry{top=25mm}
\geometry{bottom=35mm}
\geometry{left=20mm}
\geometry{right=20mm}
% ------------------------------------------------

% Graphics
\usepackage{color}
\usepackage{tabularx}
\usepackage{tikz}
\usepackage{blkarray}
\usepackage{graphicx}
% ------------------------------------------------

% Math
\usepackage{amsmath, amsfonts}
\usepackage{amssymb}
\usepackage{proof}
\usepackage{mathrsfs}
% Crossed-out symbols
% https://tex.stackexchange.com/questions/75525/how-to-write-crossed-out-math-in-latex
\usepackage[makeroom]{cancel}
\usepackage{mathtools}
% ------------------------------------------------

% Additional font sizes
% https://www.overleaf.com/learn/latex/Questions/How_do_I_adjust_the_font_size%3F
\usepackage{moresize}
% Additional colors
% https://www.overleaf.com/learn/latex/Using_colours_in_LaTeX
\usepackage{xcolor}
% \texttimes
\usepackage{textcomp}
% ------------------------------------------------

% Language
\usepackage[utf8] {inputenc}
\usepackage[T2A] {fontenc}
\usepackage[english, russian] {babel}
\usepackage{indentfirst, verbatim}
\usetikzlibrary{cd, babel}
% ------------------------------------------------

% Fonts
\usepackage{stmaryrd}
\usepackage{cmbright}
\usepackage{wasysym}
% ------------------------------------------------

% Code
% https://tex.stackexchange.com/questions/99475/how-to-invoke-latex-with-the-shell-escape-flag-in-texstudio-former-texmakerx
% Colored, requires --shell-escape compiling option
% \usepackage{minted}
% \setminted{xleftmargin=\parindent, autogobble, escapeinside=\#\#}
\usepackage{listings}
% ------------------------------------------------

% Custom envs
% https://tex.stackexchange.com/questions/371286/draw-a-horizontal-line-in-latex
\newenvironment{proof}{\subparagraph{\hspace{-1em}Решение:\newline}}{\par\noindent\rule{\textwidth}{0.4pt}}
% ------------------------------------------------

% Custom commands
\newcommand{\comb}[1]{\mathbf{#1}}
\newcommand{\step}{\rightsquigarrow}
\newcommand{\term}[1]{\mathbf{#1}}
\newcommand{\ap}{~}
\newcommand{\termdef}{\coloneqq}
\newcommand{\subst}[3]{\left[#2 \mapsto #3 \right] #1}
\newcommand{\eqbeta}{=_\beta}
\newcommand{\eqeta}{=_\eta}
\def\multiset#1#2{\ensuremath{\left(\kern-.3em\left(\genfrac{}{}{0pt}{}{#1}{#2}\right)\kern-.3em\right)}}
% ------------------------------------------------

% Head
\usepackage{fancybox,fancyhdr}
\usepackage{hyperref}
\pagestyle{fancy}
\fancyhead[R]{Максим Васильев (285800)} % TODO введите ваше имя
\fancyhead[L]{ИТМО MSE, ДМ 2023, Дз 5}
% ------------------------------------------------

% Numbering
% https://tex.stackexchange.com/questions/80113/hide-section-numbers-but-keep-numbering
\makeatletter
\renewcommand\thesubsection{Блок \@arabic\c@subsection.\hspace{-0.8em}}
\renewcommand\thesubsubsection{Задание \@arabic\c@subsection.\@arabic\c@subsubsection\hspace{-0.8em}}
% https://tex.stackexchange.com/questions/327689/numbering-subsubsections-with-letters
\renewcommand\theparagraph{\alph{paragraph})\hspace{-0.8em}}
% https://tex.stackexchange.com/questions/129208/numbering-paragraphs-in-latex
\setcounter{secnumdepth}{4}
\makeatother
% ------------------------------------------------

\begin{document}

  \begin{enumerate}
    \item \textit{(по 0,5 балла за каждый пункт)} Из колоды, в которой 52 карты, наудачу вынимаются какие-то три. Найдите вероятность того, что:
    \begin{itemize}
        \item среди них окажется ровно один туз;
        
        \textbf{Решение}:

        Число способов выбрать 3 карты из 52:
        \begin{equation}
            \binom{52}{3}.
        \end{equation}
        Благоприятный исход: вытащить один туз из четырех $\binom{4}{1}$ и любые две, но не туз $\binom{48}{2}$. Тогда вероятность будет равна:

        \textbf{Ответ}:
        $$\dfrac{\binom{4}{1} \cdot \binom{48}{2}}{\binom{52}{3}}$$

        \item среди них окажется хотя бы один туз;
        
        \textbf{Решение}:

        Решим от обратного: искомая вероятность будет равна 1 минус вероятность того, что ни одного туза не выпало (подходящих карт для этого 48), поэтому

        \textbf{Ответ}:
        $$1-\dfrac{\binom{48}{3}}{\binom{52}{3}}$$

        \item это будут тройка, семерка и туз.
        
        \textbf{Решение}:

        Благоприятный исход: вытащить одну тройку $\binom{4}{1}$, и вытащить одну семерку $\binom{4}{1}$, и вытащить одного туза $\binom{4}{1}$. Тогда искомая вероятность будет равна:

        \textbf{Ответ}:
        $$\dfrac{4^3}{\binom{52}{3}}$$

    \end{itemize}
    \item \textit{(1 балл)} В мешке лежат карточки с буквами А, Б, В, Г, а также с цифрами 1, 2,
    3, 4, 5, 6 (всего 10 карточек). Их по очереди вынимают из мешка, пока не вынут
    все. Какова вероятность того, что буквы будут появляться в алфавитном порядке,
    а цифры — в порядке возрастания? (Расположение букв относительно цифр может
    быть любым.)

    \textbf{Решение}:

    Всего возможных вариантов -- число перестановок всех элементов:
    \begin{equation}
        10!
    \end{equation}
    Если предположить, что число подходящих нам последовательностей букв и цифр равно $n$, то, если мы рассмотрим все возможные перестановки цифр ($6!$) и букв ($4!$), получим вообще все варианты:
    \begin{equation}
        6!4!n = 10! \Rightarrow n = \frac{10!}{6!4!}
    \end{equation}
    Тогда иская вероятность по определению будет равна:

    \textbf{Ответ}:
    $$\dfrac{1}{6!4!}$$

    \item \textit{(2 балла)} Из стандартной 52–карточной колоды случайно вытаскивают 5 карт. Найдите вероятность того, что эти 5 карт образуют фулл хаус (комбинация, состоящая
    из трех карт одного номинала и двух другого).

    \textbf{Решение}:

    Число способов выбрать 5 карт из 52:
    \begin{equation}
        \binom{52}{5}.
    \end{equation}
    Благоприятный исход: вытащить 3 карты одного номинала из 4 мастей и вытащить 2 карты другого номинала из 4 мастей:
    \begin{equation}
        13 \cdot \binom{4}{3} \cdot 12 \cdot \binom{4}{2}.
    \end{equation}
    Тогда иская вероятность по определению будет равна:

    \textbf{Ответ}:
    $$\dfrac{13 \cdot \binom{4}{3} \cdot 12 \cdot \binom{4}{2}}{\binom{52}{5}}$$


    \item \textit{(1 балл)} Паша и Слава загадывают по числу от 1 до 10 (все числа для них одинаково
    хороши и они выбирают их равновероятно). Независимы ли следующие события:
    A = число Паши делится на 4,
    B = сумма чисел Паши и Славы делится на 4?

    \textbf{Решение}:

    Посчитаем вероятности событий A и B. Вероятность первого события 0.2, так как нам подходят только четверка и восьмерка из 10 возможных случаев. Вероятность второго события получается из следующих соображений: всего вариантов пар $10^2$; возможные суммы: $4,8,12,16,20$; количества вариантов для каждой из сумм:
    \begin{equation}
        \{4:3,\quad8:7,\quad12:9,\quad16:5,\quad20:1\}.
    \end{equation}
    Тогда вероятность события B:
    \begin{equation}
        \frac{25}{100} = 0.25.
    \end{equation}
    Посчитаем вероятность события B, при условии A: подходят лишь следующие пары:
    \begin{equation}
        \{4:4;8,\quad 8:4;8\}.
    \end{equation}
    Поэтому условная вероятность:
    \begin{equation}
        P(B|A) = 0.04 \ne 0.05 = P(A)\cdot P(B).
    \end{equation}
    Поэтому

    \textbf{Ответ}:
    события A и B не независимы

    \item \textit{(по 0,5 балла за каждый пункт)} Четыре человека Андрей, Борис, Василий и Глеб
    становятся в очередь в случайном порядке. Найдите
    \begin{itemize}
        \item условную вероятность того, что Андрей первый, если Борис последний;
        \item условную вероятность того, что Андрей первый, если Андрей не последний;
        \item условную вероятность того, что Андрей первый, если Борис не последний;
        \item условную вероятность того, что Андрей первый, если Василий стоит в очереди
        позже Андрея;
        \item условную вероятность того, что Андрей стоит в очереди раньше Бориса, если
        известно, что Андрей стоит раньше Глеба.
    \end{itemize}
    \item \textit{(1 балл)} Два охотника отправились охотиться на кабанов. Первый из них попадает в
    цель с вероятностью 0,8, а второй — с вероятностью 0,4. Каждый из них выстрелил
    в увиденного кабана по разу, а когда они его подобрали, оказалось, что в кабана
    попала всего лишь одна пуля. Какова вероятность того, что кабана убил первый
    охотник? Второй охотник?
  \end{enumerate}

\end{document}
