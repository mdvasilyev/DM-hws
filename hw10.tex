%! suppress = Makeatletter
%! suppress = TooLargeSection
%! suppress = MissingLabel
\documentclass{article}

% Fields
\usepackage{geometry}
\geometry{top=25mm}
\geometry{bottom=35mm}
\geometry{left=20mm}
\geometry{right=20mm}
% ------------------------------------------------

% Graphics
\usepackage{color}
\usepackage{tabularx}
\usepackage{tikz}
\usepackage{blkarray}
\usepackage{graphicx}
% ------------------------------------------------

% Math
\usepackage{amsmath, amsfonts}
\usepackage{amssymb}
\usepackage{proof}
\usepackage{mathrsfs}
% Crossed-out symbols
% https://tex.stackexchange.com/questions/75525/how-to-write-crossed-out-math-in-latex
\usepackage[makeroom]{cancel}
\usepackage{mathtools}
% ------------------------------------------------

% Additional font sizes
% https://www.overleaf.com/learn/latex/Questions/How_do_I_adjust_the_font_size%3F
\usepackage{moresize}
% Additional colors
% https://www.overleaf.com/learn/latex/Using_colours_in_LaTeX
\usepackage{xcolor}
% \texttimes
\usepackage{textcomp}
% ------------------------------------------------

% Language
\usepackage[utf8] {inputenc}
\usepackage[T2A] {fontenc}
\usepackage[english, russian] {babel}
\usepackage{indentfirst, verbatim}
\usetikzlibrary{cd, babel}
% ------------------------------------------------

% Fonts
\usepackage{stmaryrd}
\usepackage{cmbright}
\usepackage{wasysym}
% ------------------------------------------------

% Code
% https://tex.stackexchange.com/questions/99475/how-to-invoke-latex-with-the-shell-escape-flag-in-texstudio-former-texmakerx
% Colored, requires --shell-escape compiling option
% \usepackage{minted}
% \setminted{xleftmargin=\parindent, autogobble, escapeinside=\#\#}
\usepackage{listings}
% ------------------------------------------------

% Custom envs
% https://tex.stackexchange.com/questions/371286/draw-a-horizontal-line-in-latex
\newenvironment{proof}{\subparagraph{\hspace{-1em}Решение:\newline}}{\par\noindent\rule{\textwidth}{0.4pt}}
% ------------------------------------------------

% Custom commands
\newcommand{\comb}[1]{\mathbf{#1}}
\newcommand{\step}{\rightsquigarrow}
\newcommand{\term}[1]{\mathbf{#1}}
\newcommand{\ap}{~}
\newcommand{\termdef}{\coloneqq}
\newcommand{\subst}[3]{\left[#2 \mapsto #3 \right] #1}
\newcommand{\eqbeta}{=_\beta}
\newcommand{\eqeta}{=_\eta}
\def\multiset#1#2{\ensuremath{\left(\kern-.3em\left(\genfrac{}{}{0pt}{}{#1}{#2}\right)\kern-.3em\right)}}
% ------------------------------------------------

% Head
\usepackage{fancybox,fancyhdr}
\usepackage{hyperref}
\pagestyle{fancy}
\fancyhead[R]{Максим Васильев (285800)} % TODO введите ваше имя
\fancyhead[L]{ИТМО MSE, ДМ 2023, Дз 10}
% ------------------------------------------------

% Numbering
% https://tex.stackexchange.com/questions/80113/hide-section-numbers-but-keep-numbering
\makeatletter
\renewcommand\thesubsection{Блок \@arabic\c@subsection.\hspace{-0.8em}}
\renewcommand\thesubsubsection{Задание \@arabic\c@subsection.\@arabic\c@subsubsection\hspace{-0.8em}}
% https://tex.stackexchange.com/questions/327689/numbering-subsubsections-with-letters
\renewcommand\theparagraph{\alph{paragraph})\hspace{-0.8em}}
% https://tex.stackexchange.com/questions/129208/numbering-paragraphs-in-latex
\setcounter{secnumdepth}{4}
\makeatother
% ------------------------------------------------

\begin{document}
\begin{enumerate}

    \item \textit{(0,5 балла)} Докажите, что дополнение несвязного графа является связным.
    
    \textbf{Решение}:

    Возьмем две вершины $u, v$, которые входят в исходный граф $G$. Если в исходном графе $G$ эти две вершины не соединены ребром, то в дополнении $\bar{G}$ они соединены, поэтому в $\bar{G}$ имеется путь из $u$ в $v$. Если в исходном графе $G$ эти две вершины соединены ребром, то они находятся в одной компоненте связности. Так как $G$ несвязный, то найдется вершина $w$ из другой компоненты связности. Поэтому в дополнении $\bar{G}$ путь из $u$ в $v$ будет проходить через эту самую вершину $w$: $u, w, v$. Поэтому, если у нас есть несвязный граф и мы строим его дополнение, то в дополнении любые две вершины всегда будут иметь путь из одной вершины в другую, то есть дополнение будет связным. $\blacksquare$

    \item \textit{(1 балл)} В стране из каждого города выходит 100 дорог и из любого города можно добраться до любого другого. Одну дорогу закрыли на ремонт. Докажите, что все равно из любого города можно добраться до любого другого.
    
    \textbf{Решение}:

    Возьмем два города $u, v$, которые соединены прямой дорогой и перекроем эту дорогу. Предположим, что после перекрытия этой дороги, больше нельзя попасть из $u$ в $v$. Это означает, что в компоненте связности, включающей, например, город $u$ степени всех вершин четные, кроме самого города $u$. Такого не может быть, потому что в любом графе должно быть четное количество вершин нечетной степени (лемма). Противоречие, поэтому даже после перекрытия дороги $u, v$, из первого города во второй все равно можно будет попасть. $\blacksquare$
    
    \item \textit{(1 балл)} Пусть $G$ --- простой граф, построенный на 9 вершинах. Предположим, что сумма степеней вершин графа $G$ больше или равна 27. Правда ли, что в таком графе обязательно существует вершина, степень которой больше или равна 4?
    
    \textit{Определение.} \textbf{Турнир} --- это ориентированный граф, полученный из неориентированного полного графа путём назначения направления каждому ребру.
    
    \item \textit{(1 балл)} Докажите, что в случае нечетных $n$ существует турнир $T$, в котором для любой вершины $x$ выполняется равенство: $outdeg(x) = indeg(x)$.
    
    \item \textit{(по 0,5 балла за пункт)} Верно ли, что два графа изоморфны, если\\
    а) у них по 8 вершин, степень каждой из которых равна 3?\\
    б) они связны, без циклов и содержат по 6 ребер?\\
    в) у них по 10 вершин, степень каждой из которых равна 9?
    
    \item \textit{(2 балла)} Рассмотрим граф, вершинами которого являются всевозможные бинарные последовательности длины $n$. Две вершины соединены ребром, если соответствующие бинарные последовательности различаются ровно в одном элементе. Докажите, что полученный граф является регулярным двудольным графом. А также подсчитайте количество вершин и ребер в этом графе.
    
\end{enumerate}
\end{document}
