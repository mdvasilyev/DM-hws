%! suppress = Makeatletter
%! suppress = TooLargeSection
%! suppress = MissingLabel
\documentclass{article}

% Fields
\usepackage{geometry}
\geometry{top=25mm}
\geometry{bottom=35mm}
\geometry{left=20mm}
\geometry{right=20mm}
% ------------------------------------------------

% Graphics
\usepackage{color}
\usepackage{tabularx}
\usepackage{tikz}
\usepackage{blkarray}
\usepackage{graphicx}
% ------------------------------------------------

% Math
\usepackage{amsmath, amsfonts}
\usepackage{amssymb}
\usepackage{proof}
\usepackage{mathrsfs}
% Crossed-out symbols
% https://tex.stackexchange.com/questions/75525/how-to-write-crossed-out-math-in-latex
\usepackage[makeroom]{cancel}
\usepackage{mathtools}
% ------------------------------------------------

% Additional font sizes
% https://www.overleaf.com/learn/latex/Questions/How_do_I_adjust_the_font_size%3F
\usepackage{moresize}
% Additional colors
% https://www.overleaf.com/learn/latex/Using_colours_in_LaTeX
\usepackage{xcolor}
% \texttimes
\usepackage{textcomp}
% ------------------------------------------------

% Language
\usepackage[utf8] {inputenc}
\usepackage[T2A] {fontenc}
\usepackage[english, russian] {babel}
\usepackage{indentfirst, verbatim}
\usetikzlibrary{cd, babel}
% ------------------------------------------------

% Fonts
\usepackage{stmaryrd}
\usepackage{cmbright}
\usepackage{wasysym}
% ------------------------------------------------

% Code
% https://tex.stackexchange.com/questions/99475/how-to-invoke-latex-with-the-shell-escape-flag-in-texstudio-former-texmakerx
% Colored, requires --shell-escape compiling option
% \usepackage{minted}
% \setminted{xleftmargin=\parindent, autogobble, escapeinside=\#\#}
\usepackage{listings}
% ------------------------------------------------

% Custom envs
% https://tex.stackexchange.com/questions/371286/draw-a-horizontal-line-in-latex
\newenvironment{proof}{\subparagraph{\hspace{-1em}Решение:\newline}}{\par\noindent\rule{\textwidth}{0.4pt}}
% ------------------------------------------------

% Custom commands
\newcommand{\comb}[1]{\mathbf{#1}}
\newcommand{\step}{\rightsquigarrow}
\newcommand{\term}[1]{\mathbf{#1}}
\newcommand{\ap}{~}
\newcommand{\termdef}{\coloneqq}
\newcommand{\subst}[3]{\left[#2 \mapsto #3 \right] #1}
\newcommand{\eqbeta}{=_\beta}
\newcommand{\eqeta}{=_\eta}
\def\multiset#1#2{\ensuremath{\left(\kern-.3em\left(\genfrac{}{}{0pt}{}{#1}{#2}\right)\kern-.3em\right)}}
% ------------------------------------------------

% Head
\usepackage{fancybox,fancyhdr}
\usepackage{hyperref}
\pagestyle{fancy}
\fancyhead[R]{Максим Васильев (285800)} % TODO введите ваше имя
\fancyhead[L]{ИТМО MSE, ДМ 2023, Дз 12}
% ------------------------------------------------

% Numbering
% https://tex.stackexchange.com/questions/80113/hide-section-numbers-but-keep-numbering
\makeatletter
\renewcommand\thesubsection{Блок \@arabic\c@subsection.\hspace{-0.8em}}
\renewcommand\thesubsubsection{Задание \@arabic\c@subsection.\@arabic\c@subsubsection\hspace{-0.8em}}
% https://tex.stackexchange.com/questions/327689/numbering-subsubsections-with-letters
\renewcommand\theparagraph{\alph{paragraph})\hspace{-0.8em}}
% https://tex.stackexchange.com/questions/129208/numbering-paragraphs-in-latex
\setcounter{secnumdepth}{4}
\makeatother
% ------------------------------------------------

\begin{document}
\begin{enumerate}
    \item \textit{(0,5 балла)} Докажите, что любое дерево является двудольным графом.
    
    \textbf{Решение}:

    Граф называется двудольным, если множество его вершин можно разбить на два непересекающихся подмножества таким образом, чтобы никакие две вершины из одного множества не были соединены ребром.

    Деревом называется простой связных граф, в котором между любыми двумя вершинами существует единственный путь.

    Покажем, что множество вершин дерева всегда можно разбить на два множества так, чтобы получился двудольный граф. Для этого зафиксируем какую-нибудь вершину дерева $v$ и отнесем ее к первому множеству. Дальше будем перебирать все вершины по порядку, и если длина пути до очередной вершины нечетна, то будем относить ее ко второму множеству, а если четна, то к первому. В итоге получим два непересекающихся множества вершин, объединение которых даст исходное множество вершин графа.

    Покажем, что любые две вершины $v_1, v_2$, обладающие одним ребром, лежат в разных подмножествах. Для этого возьмем какую-нибудь вершину $v'$ из первого множества и предположим, что вершина $v_1$ тоже лежит в первом множестве. Так как они лежат в одном подмножестве вершин, то путь от одной до другой - четный. Так как путь между любыми двумя вершинами в дереве единственный, то путь между $v'$ и $v_2$ обязательно проходит по ребру $v_1, v_2$, поэтому путь от $v'$ до $v_2$ нечетный, поэтому вершины $v_1, v_2$ лежат в разных подмножествах. $\blacksquare$

    \item \textit{(1 балл)} Полным $m$-арным деревом называется корневое дерево, у которого любая вершина, отличная от листа, имеет ровно $m$ сыновей. Предположим, что у такого дерева имеется k вершин, отличных от листа. Найдите количество листьев в таком дереве.
    \item \textit{(1 балл)} Найдите количество неизоморфных друг другу деревьев на $n \geq 4$ вершинах, диаметр которых меньше или равен трем.
    \item \textit{(1 балл)} Пусть T — дерево, в котором степень любой вершины, смежной с листом, больше или равна трем. Докажите, что в T обязательно найдется пара листьев, имеющих общего соседа.
    \item \textit{(0.5 балла)} Постройте последовательность Прюфера для данного дерева:
    \item \textit{(0.5 балла)} Постройте дерево, отвечающее последовательности Прюфера $(1, 7, 2, 2, 2, 2)$.
    \item \textit{(1.5 балла)} Подсчитайте количество всех деревьев, построенных на множестве вершин $[n]$, у которых вершина с меткой $i$ имеет степень $d_i$. Выведите отсюда формулу Кэли для количества всех деревьев на n вершинах.
\end{enumerate}
\end{document}
