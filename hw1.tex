%! suppress = Makeatletter
%! suppress = TooLargeSection
%! suppress = MissingLabel
\documentclass{article}

% Fields
\usepackage{geometry}
\geometry{top=25mm}
\geometry{bottom=35mm}
\geometry{left=20mm}
\geometry{right=20mm}
% ------------------------------------------------

% Graphics
\usepackage{color}
\usepackage{tabularx}
\usepackage{tikz}
\usepackage{blkarray}
\usepackage{graphicx}
% ------------------------------------------------

% Math
\usepackage{amsmath, amsfonts}
\usepackage{amssymb}
\usepackage{proof}
\usepackage{mathrsfs}
% Crossed-out symbols
% https://tex.stackexchange.com/questions/75525/how-to-write-crossed-out-math-in-latex
\usepackage[makeroom]{cancel}
\usepackage{mathtools}
% ------------------------------------------------

% Additional font sizes
% https://www.overleaf.com/learn/latex/Questions/How_do_I_adjust_the_font_size%3F
\usepackage{moresize}
% Additional colors
% https://www.overleaf.com/learn/latex/Using_colours_in_LaTeX
\usepackage{xcolor}
% \texttimes
\usepackage{textcomp}
% ------------------------------------------------

% Language
\usepackage[utf8] {inputenc}
\usepackage[T2A] {fontenc}
\usepackage[english, russian] {babel}
\usepackage{indentfirst, verbatim}
\usetikzlibrary{cd, babel}
% ------------------------------------------------

% Fonts
\usepackage{stmaryrd}
\usepackage{cmbright}
\usepackage{wasysym}
% ------------------------------------------------

% Code
% https://tex.stackexchange.com/questions/99475/how-to-invoke-latex-with-the-shell-escape-flag-in-texstudio-former-texmakerx
% Colored, requires --shell-escape compiling option
% \usepackage{minted}
% \setminted{xleftmargin=\parindent, autogobble, escapeinside=\#\#}
\usepackage{listings}
% ------------------------------------------------

% Custom envs
% https://tex.stackexchange.com/questions/371286/draw-a-horizontal-line-in-latex
\newenvironment{proof}{\subparagraph{\hspace{-1em}Решение:\newline}}{\par\noindent\rule{\textwidth}{0.4pt}}
% ------------------------------------------------

% Custom commands
\newcommand{\comb}[1]{\mathbf{#1}}
\newcommand{\step}{\rightsquigarrow}
\newcommand{\term}[1]{\mathbf{#1}}
\newcommand{\ap}{~}
\newcommand{\termdef}{\coloneqq}
\newcommand{\subst}[3]{\left[#2 \mapsto #3 \right] #1}
\newcommand{\eqbeta}{=_\beta}
\newcommand{\eqeta}{=_\eta}
% ------------------------------------------------

% Head
\usepackage{fancybox,fancyhdr}
\usepackage{hyperref}
\pagestyle{fancy}
\fancyhead[R]{Максим Васильев (285800)} % TODO введите ваше имя
\fancyhead[L]{ИТМО MSE, ДМ 2023, Дз 1}
% ------------------------------------------------

% Numbering
% https://tex.stackexchange.com/questions/80113/hide-section-numbers-but-keep-numbering
\makeatletter
\renewcommand\thesubsection{Блок \@arabic\c@subsection.\hspace{-0.8em}}
\renewcommand\thesubsubsection{Задание \@arabic\c@subsection.\@arabic\c@subsubsection\hspace{-0.8em}}
% https://tex.stackexchange.com/questions/327689/numbering-subsubsections-with-letters
\renewcommand\theparagraph{\alph{paragraph})\hspace{-0.8em}}
% https://tex.stackexchange.com/questions/129208/numbering-paragraphs-in-latex
\setcounter{secnumdepth}{4}
\makeatother
% ------------------------------------------------

\begin{document}

    \begin{enumerate}
        \item \textit{(1 балл)} Какое максимальное количество королей можно поместить на шахматную доску 8 x 8 так, чтобы эти короли не били друг друга?
        
        \textbf{Решение}:

        Король может ходить на любую соседнюю клетку, поэтому в квадрате размером 2 на 2 может находиться только один король, так как все клетки попарно соседние. Рассмотрим квадрат размером 4 на 4, который можно получить трансляцией квадрата 2 на 2 по осям $x$ и $y$. Квадрат 4 на 4 представляет собой 4 коробки, каждая из которых размером 2 на 2. Соответственно, в эти четыре коробки можно уместить лишь по одному королю. Добавить пятого короля в эти четыре коробки не получится, потому что по принципу Дирихле: хотя бы в одной коробке окажется два короля, а так быть не может, если мы хотим, чтобы они не били друг друга. Заметим, что трансляция позволяет гарантировать, что король никого побить не сможет, так как он фактически окружен слоем из пустых ячеек.

        Квадрат 8 на 8 можем получить трансляцией квадрата 4 на 4 по осям $x$ и $y$. Благодаря трансляции короли из соседних блоков размера 4 на 4 никак друг на друга не влияют, а для каждого квадрата 4 на 4 мы выяснили, что максимально допустимое количество королей -- это 4. Следовательно, на поле 8 x 8 максимально может находиться 16 королей так, чтобы эти короли не били друг друга.

        Вариант корректной расстановки королей на поле приведен на картинке ниже.

        \begin{center}
        \begin{tikzpicture}
            \draw[thick, ->] (0*0.5,0*0.5) -- (9*0.5,0*0.5) node[anchor=north west]{x};
            \draw[thick, ->] (0*0.5,0*0.5) -- (0*0.5,9*0.5) node[anchor=south east]{y};
            \draw[step=1*0.5cm] (0*0.5,0*0.5) grid (8*0.5,8*0.5);
            \filldraw[gray, draw=black] (0*0.5,0*0.5) rectangle (1*0.5,1*0.5);
            \filldraw[gray, draw=black] (2*0.5,0*0.5) rectangle (3*0.5,1*0.5);
            \filldraw[gray, draw=black] (4*0.5,0*0.5) rectangle (5*0.5,1*0.5);
            \filldraw[gray, draw=black] (6*0.5,0*0.5) rectangle (7*0.5,1*0.5);
            \filldraw[gray, draw=black] (0*0.5,2*0.5) rectangle (1*0.5,3*0.5);
            \filldraw[gray, draw=black] (2*0.5,2*0.5) rectangle (3*0.5,3*0.5);
            \filldraw[gray, draw=black] (4*0.5,2*0.5) rectangle (5*0.5,3*0.5);
            \filldraw[gray, draw=black] (6*0.5,2*0.5) rectangle (7*0.5,3*0.5);
            \filldraw[gray, draw=black] (0*0.5,4*0.5) rectangle (1*0.5,5*0.5);
            \filldraw[gray, draw=black] (2*0.5,4*0.5) rectangle (3*0.5,5*0.5);
            \filldraw[gray, draw=black] (4*0.5,4*0.5) rectangle (5*0.5,5*0.5);
            \filldraw[gray, draw=black] (6*0.5,4*0.5) rectangle (7*0.5,5*0.5);
            \filldraw[gray, draw=black] (0*0.5,6*0.5) rectangle (1*0.5,7*0.5);
            \filldraw[gray, draw=black] (2*0.5,6*0.5) rectangle (3*0.5,7*0.5);
            \filldraw[gray, draw=black] (4*0.5,6*0.5) rectangle (5*0.5,7*0.5);
            \filldraw[gray, draw=black] (6*0.5,6*0.5) rectangle (7*0.5,7*0.5);
        \end{tikzpicture}

        \textit{Рисунок 1 -- Расстановка королей}
        \end{center}

        \textbf{Ответ}:

        16 королей.
        \item \textit{(1 балл)} Узлы бесконечной клетчатой бумаги покрашены в два цвета. Докажите, что существуют две горизонтальные и две вертикальные прямые, на пересечениях которых лежат точки, покрашенные в один и тот же цвет.
        
        \textbf{Решение}:

        \textbf{Ответ}:

        \item \textit{(1,5 балла)} Докажите, что в любой выборке из 52 положительных целых чисел найдутся хотя бы два, у которых либо их сумма, либо их разность делится на 100.
        
        \textbf{Решение}:

        $\square$

        При делении на 100 можно получить сто остатков от деления: $0, 1, \ldots, 99$. Разобьем их на следующие множества в количестве 51:
        $$\{0\}, \{50\}, \{1, 99\}, \{2, 98\}, \{3, 97\}, \ldots \{49, 51\}$$
        Имея 52 числа, по принципу Дирихле можно заключить, что в одно множество попадет хотя бы два числа из данных 52. Рассмотрим первое $\{0\}$ и второе $\{50\}$ множества: если какие-то два числа $n_1$ и $n_2$ попадают в одно из этих множеств, то, что их разность, что их сумма тоже будут делиться на 100. Если два числа попадут в любое из множеств $\{1, 99\}, \{2, 98\}, \{3, 97\}, \ldots \{49, 51\}$, то возможны несколько случаев: если два числа $n_1$ и $n_2$ будут иметь одинаковый (в рамках коробки) остаток от деления на 100, то их разность будет делиться на 100; если два числа $n_1$ и $n_2$ будут иметь разные (в рамках коробки) остатки от деления на 100, то их сумма будет делиться на 100, потому что коробки так устроены, что в сумме они дают как раз 100.
        \begin{flushright}
            $\blacksquare$
        \end{flushright}

        \item \textit{(1 балл)} (Неравенство Бернулли) Докажите, что $(1+a)^n\geq1+na$ для любого вещественного $a>-1$ и для любого натурального $n$.
        
        \textbf{Решение}:

        $\square$

        \textit{База индукции}:
        
        $n=1$: 

        $$(1+a)^1=1+1*a>$$

        \textit{Индукционный переход}: $k\mapsto(k+1)$

        Предполагаем, что для $n=k$ неравенство верно. Рассмотрим $n=k+1$:
        $$(1+a)^{n+1}=(1+a)^n(1+a)\geq(1+na)(1+a)=1+a+na+na^2=1+(n+1)a+na^2$$
        Учитывая, что $n\geq1$ и $a^2\geq0$, можно сделать вывод, что $na^2\geq0$. Тогда
        $$(1+a)^{n+1}\geq1+(n+1)a+na^2\geq1+(n+1)a$$

        \begin{flushright}
            $\blacksquare$
        \end{flushright}

    \end{enumerate}

\end{document}
