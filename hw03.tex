%! suppress = Makeatletter
%! suppress = TooLargeSection
%! suppress = MissingLabel
\documentclass{article}

% Fields
\usepackage{geometry}
\geometry{top=25mm}
\geometry{bottom=35mm}
\geometry{left=20mm}
\geometry{right=20mm}
% ------------------------------------------------

% Graphics
\usepackage{color}
\usepackage{tabularx}
\usepackage{tikz}
\usepackage{blkarray}
\usepackage{graphicx}
% ------------------------------------------------

% Math
\usepackage{amsmath, amsfonts}
\usepackage{amssymb}
\usepackage{proof}
\usepackage{mathrsfs}
% Crossed-out symbols
% https://tex.stackexchange.com/questions/75525/how-to-write-crossed-out-math-in-latex
\usepackage[makeroom]{cancel}
\usepackage{mathtools}
% ------------------------------------------------

% Additional font sizes
% https://www.overleaf.com/learn/latex/Questions/How_do_I_adjust_the_font_size%3F
\usepackage{moresize}
% Additional colors
% https://www.overleaf.com/learn/latex/Using_colours_in_LaTeX
\usepackage{xcolor}
% \texttimes
\usepackage{textcomp}
% ------------------------------------------------

% Language
\usepackage[utf8] {inputenc}
\usepackage[T2A] {fontenc}
\usepackage[english, russian] {babel}
\usepackage{indentfirst, verbatim}
\usetikzlibrary{cd, babel}
% ------------------------------------------------

% Fonts
\usepackage{stmaryrd}
\usepackage{cmbright}
\usepackage{wasysym}
% ------------------------------------------------

% Code
% https://tex.stackexchange.com/questions/99475/how-to-invoke-latex-with-the-shell-escape-flag-in-texstudio-former-texmakerx
% Colored, requires --shell-escape compiling option
% \usepackage{minted}
% \setminted{xleftmargin=\parindent, autogobble, escapeinside=\#\#}
\usepackage{listings}
% ------------------------------------------------

% Custom envs
% https://tex.stackexchange.com/questions/371286/draw-a-horizontal-line-in-latex
\newenvironment{proof}{\subparagraph{\hspace{-1em}Решение:\newline}}{\par\noindent\rule{\textwidth}{0.4pt}}
% ------------------------------------------------

% Custom commands
\newcommand{\comb}[1]{\mathbf{#1}}
\newcommand{\step}{\rightsquigarrow}
\newcommand{\term}[1]{\mathbf{#1}}
\newcommand{\ap}{~}
\newcommand{\termdef}{\coloneqq}
\newcommand{\subst}[3]{\left[#2 \mapsto #3 \right] #1}
\newcommand{\eqbeta}{=_\beta}
\newcommand{\eqeta}{=_\eta}
\def\multiset#1#2{\ensuremath{\left(\kern-.3em\left(\genfrac{}{}{0pt}{}{#1}{#2}\right)\kern-.3em\right)}}
% ------------------------------------------------

% Head
\usepackage{fancybox,fancyhdr}
\usepackage{hyperref}
\pagestyle{fancy}
\fancyhead[R]{Максим Васильев (285800)} % TODO введите ваше имя
\fancyhead[L]{ИТМО MSE, ДМ 2023, Дз 3}
% ------------------------------------------------

% Numbering
% https://tex.stackexchange.com/questions/80113/hide-section-numbers-but-keep-numbering
\makeatletter
\renewcommand\thesubsection{Блок \@arabic\c@subsection.\hspace{-0.8em}}
\renewcommand\thesubsubsection{Задание \@arabic\c@subsection.\@arabic\c@subsubsection\hspace{-0.8em}}
% https://tex.stackexchange.com/questions/327689/numbering-subsubsections-with-letters
\renewcommand\theparagraph{\alph{paragraph})\hspace{-0.8em}}
% https://tex.stackexchange.com/questions/129208/numbering-paragraphs-in-latex
\setcounter{secnumdepth}{4}
\makeatother
% ------------------------------------------------

\begin{document}

  \begin{enumerate}
    \item \textit{(1 балл)} Докажите комбинаторно, что для всех целых $n \geq m \geq k \geq 0$ справедливо равенство
    $$\binom{n}{m} \cdot \binom{m}{k} = \binom{n}{k} \cdot \binom{n - k}{m - k}.$$
    \textbf{Решение}:

    Предположим, что у нас имеется $n$ коробок, на которые мы хотим наклеить $m$ открыток. После наклеивания открыток мы хотим положить в них $k$ подарков по одному в каждую коробку. При этом должно выполняться условие $n \geq m \geq k \geq 0$. Как мы можем это сделать?
    \begin{itemize}
      \item Выбрать $m$ коробок из $n$, на которые мы наклеим по открытке, а потом выбрать $k$ коробок из $m$ (на которых есть по открытке), куда мы положим подарки:
      \begin{equation}
        \binom{n}{m} \cdot \binom{m}{k}.
      \end{equation}
      \item Также мы могли бы сначала выбрать $k$ коробок из $n$ возможных, куда бы мы положили по подарку и наклеили по открытке, а потом на оставшиеся $n-k$ коробок без открыток наклеили бы оставшиеся $m-k$ открыток:
      \begin{equation}
        \binom{n}{k} \cdot \binom{n - k}{m - k}.
      \end{equation}
      Так как исход в обоих случаях одинаковый, можно сделать вывод, что
      \begin{equation}
        \binom{n}{m} \cdot \binom{m}{k} = \binom{n}{k} \cdot \binom{n - k}{m - k}.
      \end{equation}
    \end{itemize}
    \begin{flushright}
      $\blacksquare$
    \end{flushright}
    \item \textit{(1 балл)} Докажите комбинаторно следующее тождество:
    $$\multiset{n + 1}{k} = \sum_{i = 0}^{k} \multiset{n}{k - i}.$$
    \textbf{Решение}:

    Представим, что мы купили куртку, в которой $n + 1$ карман. Цель покупки такой куртки -- отправиться в школу, взяв с собой $k$ карандашей из ИКЕИ, разложенных по карманам. Карманы мы можем различать, а карандаши из ИКЕИ, очевидно, нет. В каждый карман можем положить от 0 до $k$ карандашей. Как мы можем уйти в школу?
    \begin{itemize}
      \item Разложить $k$ неразличимых карандашей в $n + 1$ карман, причем любой карман может вместить от 0 до $k$ карандашей:
      \begin{equation}
        \multiset{n + 1}{k}.
      \end{equation}
      \item Зафиксировать один карман, который ближе всего к сердцу и положить в него все $k$ карандашей, тогда на оставшиеся $n$ карманов останется 0 карандашей, а вариантов разместить 0 карандашей по $n$ карманам:
      \begin{equation}
        \multiset{n}{0}.
      \end{equation}
      Но есть еще варианты. Положим в карман, который ближе к сердцу $k-1$ карандаш, тогда на оставшиеся $n$ карманов придется распределить 1 карандаш. Или положим в карман, который ближе к сердцу $k-2$ карандаша, тогда на оставшиеся $n$ карманов придется распределить 2 карандаша. И так далее. Всего получится:
      \begin{equation}
        \multiset{n}{0} + \multiset{n}{1} + \ldots + \multiset{n}{k} = \sum_{i = 0}^{k} \multiset{n}{k - i}.
      \end{equation}
    \end{itemize}
    Так как исход в обоих подходах одинаковый, можно сделать вывод, что
    \begin{equation}
      \multiset{n + 1}{k} = \sum_{i = 0}^{k} \multiset{n}{k - i}.
    \end{equation}

    \begin{flushright}
      $\blacksquare$
    \end{flushright}

    \item \textit{(по 1 баллу за каждый пункт)} «Хромая ладья» может ходить только на одну клетку вверх или на одну клетку вправо. Изначально она стоит в нижнем левом углу
    клетчатой доски размером (2n + 1) × (2m + 1) клеток.
    \begin{itemize}
      \item[(a)] Сколькими способами «хромая ладья» может дойти до правого верхнего угла доски?
      
      \textbf{Решение}:

      Чтобы добраться до правого края нужно сделать $2n + 1$ шагов, а до верхнего края -- $2m + 1$. Так как ладья стоит на левой нижней клетке, всего шагов ей нужно сделать
      \begin{equation}
        2\cdot(n + m).
      \end{equation}
      Чтобы подняться на нужную высоту, необходимо ``расставить'' $2m$ шагов вверх из всех $2\cdot(n + m)$ шагов:
      \begin{equation}
        \binom{2\cdot(n + m)}{2m}.
      \end{equation}
      В остальные места однозначно ``ставятся'' шаги вправо. Или можно действовать по-другому и ``расставлять'' шаги вправо, а потом однозначно ``доставлять'' шаги вверх:
      \begin{equation}
        \binom{2\cdot(n + m)}{2n}.
      \end{equation}
      \textbf{Ответ}:
      $$\binom{2\cdot(n + m)}{2m} = \binom{2\cdot(n + m)}{2n}.$$

      \item[(b)] Хулиган Вася вырезал из этой доски центральную клетку. Сколько способов теперь?
      
      \textbf{Решение}:

      В данном случае нам нужно убрать какое-то количество решений из предыдущего ответа, потому что нам заблокировали центральную клетку. Вот как можно это сделать. Посчитаем количество путей до:
      \begin{itemize}
        \item клетки, которая правее на одну и ниже на одну, чем центральная клетка;
        \item клетки, которая левее на одну и выше на одну, чем центральная клетка;
      \end{itemize}
      а потом отдельно посчитаем, какое количество путей до правой верхней клетки. Так как у нас движение либо только вправо, либо только вверх, то решения не пересекаются, и можно будет сложить. Для наглядности приведу иллюстрацию для конкретных значений $n$ и $m$.
      \begin{center}
        \begin{tikzpicture}
            \draw[step=1*0.5cm] (0*0.5,0*0.5) grid (5*0.5,7*0.5);
            \filldraw[gray, draw=black] (2*0.5,3*0.5) rectangle (3*0.5,4*0.5);
            \node[] at (3.5*0.5,2.5*0.5) {(1)};
            \node[] at (1.5*0.5,4.5*0.5) {(2)};
        \end{tikzpicture}

        \textit{Рисунок 1 -- Доска без центральной клетки ($n = 2, m = 3$)}
      \end{center}
      От левой нижней клетки до (1):
      \begin{equation}
        \binom{n + m}{n + 1}.
      \end{equation}
      От (1) до правой верхней клетки:
      \begin{equation}
        \binom{n + m}{n - 1}.
      \end{equation}
      От левой нижней клетки до (2):
      \begin{equation}
        \binom{n + m}{m + 1}.
      \end{equation}
      От (2) до правой верхней клетки:
      \begin{equation}
        \binom{n + m}{m - 1}.
      \end{equation}
      От левой нижней до правой верхней через (1):
      \begin{equation}
        \binom{n + m}{n + 1} \cdot \binom{n + m}{n - 1}.
      \end{equation}
      От левой нижней до правой верхней через (2):
      \begin{equation}
        \binom{n + m}{m + 1} \cdot \binom{n + m}{m - 1}.
      \end{equation}
      От левой нижней до правой верхней без центральной клетки:
      \begin{equation}
        \binom{n + m}{n + 1} \cdot \binom{n + m}{n - 1} + \binom{n + m}{m + 1} \cdot \binom{n + m}{m - 1}.
      \end{equation}
      Заметим, что
      \begin{equation}
        \binom{n + m}{n + 1} = \binom{n + m}{m - 1} \qquad \binom{n + m}{m + 1} = \binom{n + m}{n - 1}.
      \end{equation}
      Преобразуем:
      \begin{eqnarray}
        \binom{n + m}{n + 1} \cdot \binom{n + m}{n - 1} + \binom{n + m}{m + 1} \cdot \binom{n + m}{m - 1} = \binom{n + m}{m - 1} \cdot \binom{n + m}{n - 1} + \binom{n + m}{n - 1} \cdot \binom{n + m}{m - 1} = \\
        2 \cdot \binom{n + m}{n - 1} \cdot \binom{n + m}{m - 1}.
      \end{eqnarray}

      \textbf{Ответ}:
      $$2 \cdot \binom{n + m}{n - 1} \cdot \binom{n + m}{m - 1}.$$

    \end{itemize}

      
  \end{enumerate}

\end{document}
