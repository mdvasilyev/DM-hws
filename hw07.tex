%! suppress = Makeatletter
%! suppress = TooLargeSection
%! suppress = MissingLabel
\documentclass{article}

% Fields
\usepackage{geometry}
\geometry{top=25mm}
\geometry{bottom=35mm}
\geometry{left=20mm}
\geometry{right=20mm}
% ------------------------------------------------

% Graphics
\usepackage{color}
\usepackage{tabularx}
\usepackage{tikz}
\usepackage{blkarray}
\usepackage{graphicx}
% ------------------------------------------------

% Math
\usepackage{amsmath, amsfonts}
\usepackage{amssymb}
\usepackage{proof}
\usepackage{mathrsfs}
% Crossed-out symbols
% https://tex.stackexchange.com/questions/75525/how-to-write-crossed-out-math-in-latex
\usepackage[makeroom]{cancel}
\usepackage{mathtools}
% ------------------------------------------------

% Additional font sizes
% https://www.overleaf.com/learn/latex/Questions/How_do_I_adjust_the_font_size%3F
\usepackage{moresize}
% Additional colors
% https://www.overleaf.com/learn/latex/Using_colours_in_LaTeX
\usepackage{xcolor}
% \texttimes
\usepackage{textcomp}
% ------------------------------------------------

% Language
\usepackage[utf8] {inputenc}
\usepackage[T2A] {fontenc}
\usepackage[english, russian] {babel}
\usepackage{indentfirst, verbatim}
\usetikzlibrary{cd, babel}
% ------------------------------------------------

% Fonts
\usepackage{stmaryrd}
\usepackage{cmbright}
\usepackage{wasysym}
% ------------------------------------------------

% Code
% https://tex.stackexchange.com/questions/99475/how-to-invoke-latex-with-the-shell-escape-flag-in-texstudio-former-texmakerx
% Colored, requires --shell-escape compiling option
% \usepackage{minted}
% \setminted{xleftmargin=\parindent, autogobble, escapeinside=\#\#}
\usepackage{listings}
% ------------------------------------------------

% Custom envs
% https://tex.stackexchange.com/questions/371286/draw-a-horizontal-line-in-latex
\newenvironment{proof}{\subparagraph{\hspace{-1em}Решение:\newline}}{\par\noindent\rule{\textwidth}{0.4pt}}
% ------------------------------------------------

% Custom commands
\newcommand{\comb}[1]{\mathbf{#1}}
\newcommand{\step}{\rightsquigarrow}
\newcommand{\term}[1]{\mathbf{#1}}
\newcommand{\ap}{~}
\newcommand{\termdef}{\coloneqq}
\newcommand{\subst}[3]{\left[#2 \mapsto #3 \right] #1}
\newcommand{\eqbeta}{=_\beta}
\newcommand{\eqeta}{=_\eta}
\def\multiset#1#2{\ensuremath{\left(\kern-.3em\left(\genfrac{}{}{0pt}{}{#1}{#2}\right)\kern-.3em\right)}}
% ------------------------------------------------

% Head
\usepackage{fancybox,fancyhdr}
\usepackage{hyperref}
\pagestyle{fancy}
\fancyhead[R]{Максим Васильев (285800)} % TODO введите ваше имя
\fancyhead[L]{ИТМО MSE, ДМ 2023, Дз 6}
% ------------------------------------------------

% Numbering
% https://tex.stackexchange.com/questions/80113/hide-section-numbers-but-keep-numbering
\makeatletter
\renewcommand\thesubsection{Блок \@arabic\c@subsection.\hspace{-0.8em}}
\renewcommand\thesubsubsection{Задание \@arabic\c@subsection.\@arabic\c@subsubsection\hspace{-0.8em}}
% https://tex.stackexchange.com/questions/327689/numbering-subsubsections-with-letters
\renewcommand\theparagraph{\alph{paragraph})\hspace{-0.8em}}
% https://tex.stackexchange.com/questions/129208/numbering-paragraphs-in-latex
\setcounter{secnumdepth}{4}
\makeatother
% ------------------------------------------------

\begin{document}
\begin{enumerate}

    \item \textit{(0,5 балла)} Вероятность попадания в цель при одном выстреле равна 0,001. Для поражения цели необходимо не менее двух попаданий. Произведено 5000 выстрелов. Найдите вероятность поражения цели.
    
    \item \textit{(1 балл)} (Продолжение задачи Банаха о спичечных коробках). Некто носит с собой два коробка спичек А и Б, в которых первоначально было $M$ и $N$ спичек соответственно. Когда ему нужна спичка, он берет ее из коробка А с вероятностью $p$ или из коробка Б с вероятностью $1-p$. Найдите вероятность того, что когда математик вынет в первый раз пустой коробок, в другом будет $r$ спичек.
    
    \item \textit{(по 1 баллу за пункт)}  В схеме Бернулли вероятность успеха равна $p$, а вероятность неудачи $q = 1-p$. Найдите вероятность того, что:\\
    a) цепочка НН (две неудачи подряд) появится раньше цепочки НУ (неудача и успех подряд);\\
    b) цепочка НН появится раньше цепочки УН.
    
    \item \textit{(1 балл)} Два человека независимо друг от друга подбрасывают монету по $n$ раз каждый. Докажите, что вероятность того, что они наберут одинаковое число гербов, совпадает с вероятностью того, что у них в сумме будет $n$ гербов.
    
    \item \textit{(1 балл)} Паша 100 раз кинул мяч в баскетбольное кольцо через всю площадку. Известно, что вероятность попасть таким броском равняется 0,03. Найдите приближенно вероятность того, что Паша попал не менее пяти раз.
    
    \item \textit{(1 балл)} Правильный кубик подбрасывают 18 000 раз. Найти приближённо вероятность того, что шестёрка выпадет не менее 2900, но не более 3050 раз.
    
    \item \textit{(1,5 балла)} В жюри, состоящем из нечетного числа $n = 2m + 14$ членов, каждый независимо от других принимает правильное решение с вероятностью $p = 0,7$. Каково минимальное число членов жюри, при котором решение, принятое большинством голосов, будет справедливым с вероятностью не меньшей, чем 0,99?
    
\end{enumerate}

  

\end{document}

