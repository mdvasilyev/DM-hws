%! suppress = Makeatletter
%! suppress = TooLargeSection
%! suppress = MissingLabel
\documentclass{article}

% Fields
\usepackage{geometry}
\geometry{top=25mm}
\geometry{bottom=35mm}
\geometry{left=20mm}
\geometry{right=20mm}
% ------------------------------------------------

% Graphics
\usepackage{color}
\usepackage{tabularx}
\usepackage{tikz}
\usepackage{blkarray}
\usepackage{graphicx}
% ------------------------------------------------

% Math
\usepackage{amsmath, amsfonts}
\usepackage{amssymb}
\usepackage{proof}
\usepackage{mathrsfs}
% Crossed-out symbols
% https://tex.stackexchange.com/questions/75525/how-to-write-crossed-out-math-in-latex
\usepackage[makeroom]{cancel}
\usepackage{mathtools}
% ------------------------------------------------

% Additional font sizes
% https://www.overleaf.com/learn/latex/Questions/How_do_I_adjust_the_font_size%3F
\usepackage{moresize}
% Additional colors
% https://www.overleaf.com/learn/latex/Using_colours_in_LaTeX
\usepackage{xcolor}
% \texttimes
\usepackage{textcomp}
% ------------------------------------------------

% Language
\usepackage[utf8] {inputenc}
\usepackage[T2A] {fontenc}
\usepackage[english, russian] {babel}
\usepackage{indentfirst, verbatim}
\usetikzlibrary{cd, babel}
% ------------------------------------------------

% Fonts
\usepackage{stmaryrd}
\usepackage{cmbright}
\usepackage{wasysym}
% ------------------------------------------------

% Code
% https://tex.stackexchange.com/questions/99475/how-to-invoke-latex-with-the-shell-escape-flag-in-texstudio-former-texmakerx
% Colored, requires --shell-escape compiling option
% \usepackage{minted}
% \setminted{xleftmargin=\parindent, autogobble, escapeinside=\#\#}
\usepackage{listings}
% ------------------------------------------------

% Custom envs
% https://tex.stackexchange.com/questions/371286/draw-a-horizontal-line-in-latex
\newenvironment{proof}{\subparagraph{\hspace{-1em}Решение:\newline}}{\par\noindent\rule{\textwidth}{0.4pt}}
% ------------------------------------------------

% Custom commands
\newcommand{\comb}[1]{\mathbf{#1}}
\newcommand{\step}{\rightsquigarrow}
\newcommand{\term}[1]{\mathbf{#1}}
\newcommand{\ap}{~}
\newcommand{\termdef}{\coloneqq}
\newcommand{\subst}[3]{\left[#2 \mapsto #3 \right] #1}
\newcommand{\eqbeta}{=_\beta}
\newcommand{\eqeta}{=_\eta}
\def\multiset#1#2{\ensuremath{\left(\kern-.3em\left(\genfrac{}{}{0pt}{}{#1}{#2}\right)\kern-.3em\right)}}
% ------------------------------------------------

% Head
\usepackage{fancybox,fancyhdr}
\usepackage{hyperref}
\pagestyle{fancy}
\fancyhead[R]{Максим Васильев (285800)} % TODO введите ваше имя
\fancyhead[L]{ИТМО MSE, ДМ 2023, Дз 8}
% ------------------------------------------------

% Numbering
% https://tex.stackexchange.com/questions/80113/hide-section-numbers-but-keep-numbering
\makeatletter
\renewcommand\thesubsection{Блок \@arabic\c@subsection.\hspace{-0.8em}}
\renewcommand\thesubsubsection{Задание \@arabic\c@subsection.\@arabic\c@subsubsection\hspace{-0.8em}}
% https://tex.stackexchange.com/questions/327689/numbering-subsubsections-with-letters
\renewcommand\theparagraph{\alph{paragraph})\hspace{-0.8em}}
% https://tex.stackexchange.com/questions/129208/numbering-paragraphs-in-latex
\setcounter{secnumdepth}{4}
\makeatother
% ------------------------------------------------

\begin{document}
\begin{enumerate}

\item \textit{(0,5 балла)} Найдите функцию распределения числа попаданий мячом в баскетбольную корзину при трех бросках, если вероятность попадания каждый раз равна 0,8.

\textbf{Решение}:

Функция распределения по определению:
\begin{equation}
    F_x(t) = \mathbb{P}(x \leq t).
\end{equation}
В данном случае функция распределения будет задана кусочной функцией. Вычислим отдельно. Вероятность попасть ровно 0 раз при трех последовательных бросках:
\begin{equation}
    \mathbb{P}(0) = (1-0.8)^3 = 0.008.
\end{equation}
Вероятность попасть один раз из трех бросков по схеме Бернулли:
\begin{equation}
    \mathbb{P}(1) = \binom{3}{1}\cdot 0.8^1 \cdot (1-0.8)^2 = 0.096.
\end{equation}
Вероятность попасть два раза из трех бросков по схеме Бернулли:
\begin{equation}
    \mathbb{P}(2) = \binom{3}{2}\cdot 0.8^2 \cdot (1-0.8)^1 = 0.384.
\end{equation}
Вероятность попасть все три раза из трех бросков по схеме Бернулли:
\begin{equation}
    \mathbb{P}(3) = 0.8^3 = 0.512.
\end{equation}
Поэтому фукнция распределения будет задаваться как:

\textbf{Ответ}:
\begin{equation}
    F_x(t) = \begin{cases}
        0 \quad \qquad  t \in (-\infty, 0) \\
        0.008 \quad\, t \in [0, 1) \\
        0.104 \quad\, t \in [1, 2) \\
        0.488 \quad\, t \in [2, 3) \\
        1 \quad \qquad  t \in [3, +\infty) \\
    \end{cases}
\end{equation}

\item \textit{(0,5 балла)} Вычислите математическое ожидание и дисперсию величины $X$, равной числу выпавших очков на игральной кости.

\textbf{Решение}:

Считаем, что кубик честный и вероятность выпадение каждого значения одинаковая и равна $1/6$.

По определению математическое ожидание:
\begin{equation}
    \mathbb{E}X = \sum_{i=1}^{6} x_i \cdot p_i = p \sum_{i=1}^{6} i = 1/6 \cdot 21 = 7/2.
\end{equation}
По определению дисперсия:
\begin{equation}
    \mathbb{D}X = \mathbb{E}(X-\mathbb{E}X)^2 = \mathbb{E}X^2 - (\mathbb{E}X)^2.
\end{equation}
Для этого посчитаем $\mathbb{E}X^2$ с учетом того, что вероятность выбить квадрат такая же, как и само число:
\begin{equation}
    \mathbb{E}X^2 = \sum_{i=1}^{6} (x_i)^2 \cdot p_i = p \sum_{i=1}^{6} i^2 = 1/6 \cdot 91 = 91/6.
\end{equation}
Тогда дисперсия:
\begin{equation}
    \mathbb{D}X = \mathbb{E}X^2 - (\mathbb{E}X)^2 = 91/6 - (21/6)^2 = 35/12.
\end{equation}

\textbf{Ответ}:
$\mathbb{E}X = 7/2; \quad \mathbb{D}X = 35/12$

\item \textit{(1 балл)} Случайная величина $X$ имеет плотность распределения вида
$$f(x)=
\begin{cases}
   a(x-1)^2, &\text{при } 1\leq x\leq5,\\
   0, &\text{иначе}.
 \end{cases}$$
Вычислите константу $a$ и определите вероятность того, что $3\leq X<4$.

\textbf{Решение}:

Константа определяется из условия, что интеграл по всем значениям равен 1:
\begin{eqnarray}
    1 = \int_{1}^{5} a(x-1)^2 dx = a\left(\int_{1}^{5}x^2 dx - 2\int_{1}^{5}x dx + \int_{1}^{5}1 dx\right) = \\
    a\left(\frac{1}{3}x^3|_1^5 - x^2|_1^5 + x|_1^5\right) = a \frac{64}{3} \Rightarrow a = \frac{3}{64}.
\end{eqnarray}
Поэтому плотность распределения равна:
\begin{equation}
    f(x)= \begin{cases}
        \frac{3}{64}(x-1)^2, &\text{при } 1\leq x\leq5,\\
        0, &\text{иначе}.
 \end{cases}
\end{equation}
Вероятность того, что $3\leq X<4$:
\begin{eqnarray}
    \mathbb{P}(3\leq X<4) = \frac{3}{64}\int_{3}^{4}(x-1)^2 dx = \frac{3}{64}\left(\frac{1}{3}x^3|_3^4 - x^2|_3^4 + x|_3^4\right) = \frac{3 \cdot 19}{64 \cdot 3} = \frac{19}{64}.
\end{eqnarray}

\textbf{Ответ}:
$a = \frac{3}{64}; \quad \mathbb{P}(3\leq X<4) = \frac{19}{64}$

\item \textit{(2 балла)} Случайная величина $X$ задается плотностью распределения
$$f(x)=
\begin{cases}
   2x-4, &\text{при } 2\leq x\leq3,\\
   0, &\text{иначе}.
 \end{cases}$$
Вычислите функцию распределения $F(x)$, вероятность $\mathbb{P}(2.5<X<3.5)$, $\mathbb{E}X$ и $\mathbb{D}X$.

\textbf{Решение}:

Функция распределения вычисляется как:
\begin{equation}
    F(t) = \begin{cases}
        0, &\text{при } t < 2,\\
        \int_{2}^{t}(2x-4)dx, &\text{при } 2\leq t\leq3,\\
        1, &\text{при } t > 3.
    \end{cases}
\end{equation}
Посчитаем интеграл:
\begin{equation}
    \int_{2}^{t}(2x-4)dx = 2\int_{2}^{t}xdx - 4\int_{2}^{t}dx = x^2 |_2^t - 4 x|_2^t = t^2 - 4 - 4t + 8 = (t-2)^2.
\end{equation}
Поэтому фукнция распределения:
\begin{equation}
    F(t) = \begin{cases}
        0, &\text{при } t < 2,\\
        (t-2)^2, &\text{при } 2\leq t\leq3,\\
        1, &\text{при } t > 3.
    \end{cases}
\end{equation}
Вероятность $\mathbb{P}(2.5<X<3.5)$ вычисляется как:
\begin{equation}
    \mathbb{P}(2.5<X<3.5) = F(3.5) - F(2.5) = 1 - (2.5-2)^2 = 0.75.
\end{equation}
Математическое ожидание вычисляется как:
\begin{equation}
    \mathbb{E}X = \int_{2}^{3}x(2x-4)dx = 2\int_{2}^{3}x^2 dx - 4\int_{2}^{3}xdx = \frac{2}{3} x^3|_2^3 - 2 x^2|_2^3 = \frac{2}{3} (3^3-2^3) - 2 (3^2-2^2) = \frac{8}{3}.
\end{equation}
Дисперсия вычисляется как:
\begin{eqnarray}
    \mathbb{D}X = \mathbb{E}X^2 - (\mathbb{E}X)^2 = \int_{2}^{3}x^2(2x-4)dx - \left(\frac{8}{3}\right)^2 = 2\int_{2}^{3}x^3dx - 4\int_{2}^{3}x^2dx - \left(\frac{8}{3}\right)^2 = \\
    \frac{1}{2} x^4|_2^3 - \frac{4}{3} x^3|_2^3 - \left(\frac{8}{3}\right)^2 = \frac{1}{2} (3^4 - 2^4) - \frac{4}{3} (3^3 - 2^3) - \left(\frac{8}{3}\right)^2 = \frac{1}{18}.
\end{eqnarray}

\textbf{Ответ}:
$F(t) = \begin{cases}
    0, &\text{при } t < 2,\\
    (t-2)^2, &\text{при } 2\leq t\leq3,\\
    1, &\text{при } t > 3.
\end{cases} \qquad \mathbb{P}(2.5<X<3.5) = 0.75 \qquad \mathbb{E}X = \dfrac{8}{3} \qquad \mathbb{D}X = \dfrac{1}{18}$

\item \textit{(2 балла)} Есть правильный жетон, у которого на одной стороне стоит цифра 2, а на другой --- 0, и есть правильный кубик, у которого на противоположных гранях написаны цифры 1, 2 и 3 соответственно. Жетон и кубик бросаются на стол. Пусть $X$ --- случайная величина, равная сумме очков на жетоне и кубике. Постройте закон распределения величины $X$ и вычислите $\mathbb{E}X$ и $\mathbb{D}X$.

\textbf{Решение}:

Множества исходов для жетона и кубика:
\begin{equation}
    \text{Coin} = \{0, 2\} \qquad \text{Cube} = \{1, 2, 3\}
\end{equation}
Возможные значения случайной величины $X$:
\begin{equation}
    X = \{1, 2, 3, 4, 5\}
\end{equation}
Рассмотрим вероятности:
\begin{equation}
    \begin{cases}
        \mathbb{P}(1) = \frac{1}{2} \cdot \frac{1}{3} = \frac{1}{6} \\
        \mathbb{P}(2) = \frac{1}{2} \cdot \frac{1}{3} = \frac{1}{6} \\
        \mathbb{P}(3) = \frac{1}{2} \cdot \frac{1}{3} + \frac{1}{2} \cdot \frac{1}{3} = \frac{1}{3} \\
        \mathbb{P}(4) = \frac{1}{2} \cdot \frac{1}{3} = \frac{1}{6} \\
        \mathbb{P}(5) = \frac{1}{2} \cdot \frac{1}{3} = \frac{1}{6}
    \end{cases}
\end{equation}
Все кроме $X=3$ строятся однозначно. Тройку можно получить $\{0, 3\}$ или $\{1, 2\}$. Функция распределения:
\begin{equation}
    F(t) = \begin{cases}
        0, &\text{при } t \in (-\infty, 1),\\
        1/6, &\text{при } t \in [1, 2),\\
        2/6, &\text{при } t \in [2, 3),\\
        4/6, &\text{при } t \in [3, 4),\\
        5/6, &\text{при } t \in [4, 5),\\
        1, &\text{при } t \in [5, +\infty).
    \end{cases}
\end{equation}
По определению математическое ожидание:
\begin{equation}
    \mathbb{E}X = \sum_{i=1}^{5} x_i \cdot p_i = 1 \cdot \frac{1}{6} + 2 \cdot \frac{1}{6} + 3 \cdot \frac{2}{6} + 4 \cdot \frac{1}{6} + 5 \cdot \frac{1}{6} = \frac{1+2+6+4+5}{6} = 3.
\end{equation}
По определению дисперсия:
\begin{equation}
    \mathbb{D}X = \mathbb{E}X^2 - (\mathbb{E}X)^2.
\end{equation}
Вычислим $\mathbb{E}X^2$ с учетом того, что вероятность квадрата случайной величины $X$ совпадает с вероятностью самой случайной величины $X$:
\begin{equation}
    \mathbb{E}X^2 = \sum_{i=1}^{5} x_i^2 \cdot p_i = 1 \cdot \frac{1}{6} + 4 \cdot \frac{1}{6} + 9 \cdot \frac{2}{6} + 16 \cdot \frac{1}{6} + 25 \cdot \frac{1}{6} = \frac{1+4+18+16+25}{6} = \frac{32}{3}
\end{equation}
Тогда дисперсия:
\begin{equation}
    \mathbb{D}X = \mathbb{E}X^2 - (\mathbb{E}X)^2 = \frac{32}{3} - 9 = \frac{5}{3}.
\end{equation}

\textbf{Ответ}:
$F(t) = \begin{cases}
    0, &\text{при } t \in (-\infty, 1),\\
    1/6, &\text{при } t \in [1, 2),\\
    2/6, &\text{при } t \in [2, 3),\\
    4/6, &\text{при } t \in [3, 4),\\
    5/6, &\text{при } t \in [4, 5),\\
    1, &\text{при } t \in [5, +\infty).
\end{cases} \qquad \mathbb{E}X = 3 \qquad \mathbb{D}X = \frac{5}{3}$

% \item \textit{(2 балла)} Случайная величина X имеет плотность распределения $f_X(x)=e^{-x}$, $x\geq0$. Какое распределение у случайной величины $Y=1-e^{-X}$.

% \item \textit{(3 балла)} Пусть $X$ --- случайная величина, у которой функция распределения $F$ непрерывна и строго возрастает. Какое распределение имеет случайная величина $Y = F(X)$?

\end{enumerate}
\end{document}
