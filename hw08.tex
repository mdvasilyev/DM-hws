%! suppress = Makeatletter
%! suppress = TooLargeSection
%! suppress = MissingLabel
\documentclass{article}

% Fields
\usepackage{geometry}
\geometry{top=25mm}
\geometry{bottom=35mm}
\geometry{left=20mm}
\geometry{right=20mm}
% ------------------------------------------------

% Graphics
\usepackage{color}
\usepackage{tabularx}
\usepackage{tikz}
\usepackage{blkarray}
\usepackage{graphicx}
% ------------------------------------------------

% Math
\usepackage{amsmath, amsfonts}
\usepackage{amssymb}
\usepackage{proof}
\usepackage{mathrsfs}
% Crossed-out symbols
% https://tex.stackexchange.com/questions/75525/how-to-write-crossed-out-math-in-latex
\usepackage[makeroom]{cancel}
\usepackage{mathtools}
% ------------------------------------------------

% Additional font sizes
% https://www.overleaf.com/learn/latex/Questions/How_do_I_adjust_the_font_size%3F
\usepackage{moresize}
% Additional colors
% https://www.overleaf.com/learn/latex/Using_colours_in_LaTeX
\usepackage{xcolor}
% \texttimes
\usepackage{textcomp}
% ------------------------------------------------

% Language
\usepackage[utf8] {inputenc}
\usepackage[T2A] {fontenc}
\usepackage[english, russian] {babel}
\usepackage{indentfirst, verbatim}
\usetikzlibrary{cd, babel}
% ------------------------------------------------

% Fonts
\usepackage{stmaryrd}
\usepackage{cmbright}
\usepackage{wasysym}
% ------------------------------------------------

% Code
% https://tex.stackexchange.com/questions/99475/how-to-invoke-latex-with-the-shell-escape-flag-in-texstudio-former-texmakerx
% Colored, requires --shell-escape compiling option
% \usepackage{minted}
% \setminted{xleftmargin=\parindent, autogobble, escapeinside=\#\#}
\usepackage{listings}
% ------------------------------------------------

% Custom envs
% https://tex.stackexchange.com/questions/371286/draw-a-horizontal-line-in-latex
\newenvironment{proof}{\subparagraph{\hspace{-1em}Решение:\newline}}{\par\noindent\rule{\textwidth}{0.4pt}}
% ------------------------------------------------

% Custom commands
\newcommand{\comb}[1]{\mathbf{#1}}
\newcommand{\step}{\rightsquigarrow}
\newcommand{\term}[1]{\mathbf{#1}}
\newcommand{\ap}{~}
\newcommand{\termdef}{\coloneqq}
\newcommand{\subst}[3]{\left[#2 \mapsto #3 \right] #1}
\newcommand{\eqbeta}{=_\beta}
\newcommand{\eqeta}{=_\eta}
\def\multiset#1#2{\ensuremath{\left(\kern-.3em\left(\genfrac{}{}{0pt}{}{#1}{#2}\right)\kern-.3em\right)}}
% ------------------------------------------------

% Head
\usepackage{fancybox,fancyhdr}
\usepackage{hyperref}
\pagestyle{fancy}
\fancyhead[R]{Максим Васильев (285800)} % TODO введите ваше имя
\fancyhead[L]{ИТМО MSE, ДМ 2023, Дз 8}
% ------------------------------------------------

% Numbering
% https://tex.stackexchange.com/questions/80113/hide-section-numbers-but-keep-numbering
\makeatletter
\renewcommand\thesubsection{Блок \@arabic\c@subsection.\hspace{-0.8em}}
\renewcommand\thesubsubsection{Задание \@arabic\c@subsection.\@arabic\c@subsubsection\hspace{-0.8em}}
% https://tex.stackexchange.com/questions/327689/numbering-subsubsections-with-letters
\renewcommand\theparagraph{\alph{paragraph})\hspace{-0.8em}}
% https://tex.stackexchange.com/questions/129208/numbering-paragraphs-in-latex
\setcounter{secnumdepth}{4}
\makeatother
% ------------------------------------------------

\begin{document}
\begin{enumerate}

\item \textit{(0,5 балла)} Найдите функцию распределения числа попаданий мячом в баскетбольную корзину при трех бросках, если вероятность попадания каждый раз равна 0,8. 

\item \textit{(0,5 балла)} Вычислите математическое ожидание и дисперсию величины $X$, равной числу выпавших очков на игральной кости.

\item \textit{(1 балл)} Случайная величина $X$ имеет плотность распределения вида
$$f(x)=
\begin{cases}
   a(x-1)^2, &\text{при } 1\leq x\leq5,\\
   0, &\text{иначе}.
 \end{cases}$$
Вычислите константу $a$ и определите вероятность того, что $3\leq X<4$.

\item \textit{(2 балла)} Случайная величина $X$ задается плотностью распределения
$$f(x)=
\begin{cases}
   2x-4, &\text{при } 2\leq x\leq3,\\
   0, &\text{иначе}.
 \end{cases}$$
Вычислите функцию распределения $F(x)$, вероятность $\mathbb{P}(2.5<X<3.5)$, $\mathbb{E}X$ и $\mathbb{D}X$.

\item \textit{(2 балла)} Есть правильный жетон, у которого на одной стороне стоит цифра 2, а на другой --- 0, и есть правильный кубик, у которого на противоположных гранях написаны цифры 1, 2 и 3 соответственно. Жетон и кубик бросаются на стол. Пусть $X$ --- случайная величина, равная сумме очков на жетоне и кубике. Постройте закон распределения величины $X$ и вычислите $\mathbb{E}X$ и $\mathbb{D}X$.

% \item \textit{(2 балла)} Случайная величина X имеет плотность распределения $f_X(x)=e^{-x}$, $x\geq0$. Какое распределение у случайной величины $Y=1-e^{-X}$.

% \item \textit{(3 балла)} Пусть $X$ --- случайная величина, у которой функция распределения $F$ непрерывна и строго возрастает. Какое распределение имеет случайная величина $Y = F(X)$?

\end{enumerate}
\end{document}
